%% Advanced fields & particles final report
%% Topic : YM instantons
%% Author : Seongmin Kim, Taeyoon Kim
%% Date : 2020. 06. 15
%% 폭 넓게 쓰라고
\documentclass{article}
\usepackage{graphicx}
\usepackage{amsmath}
\usepackage{amssymb}
\usepackage{slashed}
\usepackage{hyperref}
\usepackage{tikz}
\usepackage{pgfplots}
\usepackage{subcaption}


\title{Yang-Mills Instantons}
\author{Seongmin Kim, Taeyoon Kim}
\date{\today}

%% 폭 넓게 하라고
\usepackage{geometry}
\geometry{
    a4paper,
    total={170mm,257mm},
    left=20mm,
    top=20mm,
}

\begin{document}

\maketitle

\section{Introduction : What is an Instanton?}

%% 줄글 형식
%% Instanton is a solution to the classical field equations of motion in Euclidean space.
%% Instanton, a solutions of Euclidean EL equation are localized in (Euclidean) space and time, and have finite (Euclidean) action.
%% One way to understand instantons is to consider them as a way to evaluate tunneling events in the path integral formulation of quantum field theory.
%% Instantons are important in understanding non-perturbative effects and tunneling between vacua in quantum field theory.
%% Instantons only appear in Euclidean space, and they are not solutions to the equations of motion in Minkowski space. So, they are not physical particles or fields, in Real spacetime.
%% 너가 유연하게 introduction을 쓰라고

What is instanton? Instanton is a solution to the classical field equations of motion in Euclidean space. Instanton solutions of Euclidean EL equation are localized in (Euclidean) space and time, and have finite (Euclidean) action. This is why it is called instanton.
Instantons only appear in Euclidean space, and they are not solutions to the equations of motion in Minkowski space. So, they are not physical particles or fields, in Real spacetime. So, why do we care about instantons? 
The answer is that we essentially need wick rotation and Euclidean space to perform path integral calculations in (Minkowski) quantum field theory. 
So even we are interested in Minkowski space, we should consider instantons, which are classical solutions in Euclidean space.
In detail, how do instantons affect the QFT in Minkowski space? We will discuss how the instanton effect appears in the path integral formulation, and how it is related to various physical phenomena. 


\section{Instanton effect in Path Integral : a Toy Model}

%% Instanton effect in path integral
\subsection{$0+1$ Dimensional Toy Model : QM with a double well potential}

Let's consider a simple toy model, a quantum mechanical system with a double well potential. The action of the system is given by
\begin{equation}
    S = \int dt \left( \frac{1}{2} \dot{x}^2 - V(x) \right)
\end{equation}
where the potential is given by $V(x) = \frac{1}{4}x^4 - \frac{1}{2}x^2$.
The classical Vaccume structure of the system is shown in Fig. \ref{fig:doublewell}.
\begin{figure}[h]
    \centering
    %\includegraphics[width=0.5\textwidth]{doublewell.png}
    \caption{The vaccume structure of the double well potential, there are two minima at $x = \pm 1$.}
    \label{fig:doublewell}
\end{figure}




\subsection{Path Integral Formulation and Wick Rotation}


When we consider the path integral of the system, the path integral is given by $Z = \int \mathcal{D}x e^{iS[x]}$.
But, this integral doesn't converge well, so we need to consider the path integral in Euclidean space, $Z = \int \mathcal{D}x e^{-S_E[x]}$, where $S_E[x] = \int dt \left( \frac{1}{2} \dot{x}^2 + V(x) \right)$.
And to evaluate the original path integral, we just need to wick rotate the time variable, $t \rightarrow -i\tau$.


\subsection{Instanton Solutions in Euclidean Space}


The instanton solution of the system is given by the bounce solution, which is a solution to the Euclidean equation of motion, $\delta S_E = 0$.
The bounce solution is a solution that interpolates between the two minima of the potential, and it is localized in Euclidean time.
The action of the bounce solution is finite, and it can be evaluated by the Euclidean action of the bounce solution, $S_{\text{bounce}} = \int d\tau \left( \frac{1}{2} \dot{x}^2 + V(x) \right)$.




\subsection{Instanton Contribution to the Path Integral}

The path integral of the system is given by the sum of all possible paths. In the semiclassical limit, the path integral is dominated by the classical solutions, (i.e., the instanton solutions) and the small fluctuations around them.
So the full path integral is given by the sum of all instanton contributions, with small fluctuations around them.
\begin{equation}
    Z = \int \mathcal{D}x e^{-S[x]} = \sum_{\text{instantons}} e^{-S_{\text{inst}}} \int \mathcal{D}\delta x e^{-S_{\text{fluct}}[\delta x]}
\end{equation}
where $S_{\text{inst}}$ is the action of the instanton solution, and $S_{\text{fluct}}$ is the action of the small fluctuations around the instanton solution.
This is how the instanton effect appears in the path integral formulation of quantum field theory. Normally, the instanton effect is not easy to calculate, but in this toy model, we can calculate the instanton effect by hand, with the bounce solution.
To evaluate the path integral, we need to consider all the possible classical paths.
Any classical path between the two minima of the potential can be considered as a composition of $n$ single bounce solutions, at time $t_1, ... t_n$, and the instanton action of this classical path is given by $S = \sum_{i=1}^n S_{\text{inst}}=nS_{\text{inst}}$.

So, the Euclidean path integral, from $x = -1$ to $x = 1$, is given by the sum as :
\begin{equation}
    \langle x = 1 | e^{-HT} | x = -1 \rangle = \sum_{n \ \text{odd}} e^{-nS_{\text{inst}}} \int_{0}^T \text{d}t_1 \int_{t_1}^T \text{d}t_2 \cdots \int_{t_{n-1}}^T \text{d}t_n \int \mathcal{D}\delta x e^{-S_{\text{fluct}}}
\end{equation}

The path integral of the fluctuation can be factorized into $n+1$ parts, which are the fluctuation path integrals around each instanton solution, and one for the fluctuation around the classical ground state.
Using this factorization, we can approximated, $\int \mathcal{D}\delta x e^{-S_{\text{fluct}}} \approx \left( \int \mathcal{D}\delta x_0 e^{-S_{\text{fluct}0}} \right)^n e^{-T/2}=K^n e^{-T/2}$
Finaly, we get
\begin{equation}
    \langle x = 1 | e^{-HT} | x = -1 \rangle = e^{-T/2} \sum_{n \ \text{odd}} \frac{T^n}{n!} K^n e^{-nS_{\text{inst}}}= e^{-T/2}  \text{sinh}(KT e^{-S_{\text{inst}}})
\end{equation}

So this is the matrix element of $e^{-HT}$ from $x = -1$ to $x = 1$, using the path integral including instanton solution. 
To evaluate the propagator in Minkowski space, one cas just change the time variable back to Minkowski time, $T \rightarrow -iT$.
Then, the propagator is given by $\langle x = 1 | e^{-itH} | x = -1 \rangle = e^{-iT/2}  \text{sin}(KT e^{-S_{\text{inst}}})$.

\subsection{Tunneling Amplitude, Instanton Action and WKB Approximation}

In Hamiltonian formalism, the oscillatory behavior of the propagator
\begin{equation}
    \langle x = 1 | e^{-itH} | x = -1 \rangle = e^{-iT/2}  \text{sin}(KT e^{-S_{\text{inst}}})
\end{equation} 
can be understood due to the tunneling amplitude between the two minima of the potential.

The tunneling amplitude, $\langle x=1 | H | x=-1 \rangle $, can be evaluated with the WKB approximation, which is given by $\langle x=1 | H | x=-1 \rangle = Ke^{-S_{\text{inst}}}$, where $K$ is a constant prefactor.
So the instanton is a essential ingredient to evaluate the tunneling amplitude in the path integral formulation of quantum mechanics and quantum field theory.
This is how the instanton effect appears in the path integral formulation of quantum field theory. The instanton effect is important in understanding non-perturbative effects like tunneling between vacua in quantum field theory.
\section{Yang-Mills Instanton}

\subsection{classical Yang-Mills theory: action and equations of motion}


\subsection{Hodge decomposition and self-dual equation}


\subsection{Instanton action of (anti-)self-dual solutions}


\subsection{Explicit construction of instanton solutions, and the Moduli space}



\section{Effects of Instantons in real world physics}

\subsection{Instanton effects in QCD}


\subsection{Vaccume structure of QCD and Tunneling effects}


\subsection{Easy analogy: QM on a circle}


\subsection{$\theta$-vacuum and $\theta$-angle}


\subsection{The $U_A(1)$ problem and the $\eta'$ meson}


\subsection{Chiral Anomaly and $\theta$-vacuum}


\section{Instanton effects in Axion phtsics}

\section{References}





\end{document}

