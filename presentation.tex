%%use beamer to create a presentation
%%high energy physics class final project
%%topic : YM instantons
\documentclass[10pt]{beamer}
\usepackage{graphicx}
\usepackage{amsmath}
\usepackage{amssymb}
\usepackage{slashed}
\usepackage{hyperref}
\usepackage{tikz}
\usepackage{pgfplots}
\usepackage{subcaption}
\usepackage{tikz-feynman}
\usepackage{tikz-3dplot}
\usepackage{tikz-cd}
\usepackage{physics}
\usepackage{cancel}
\usepackage{color}
\usepackage{listings}

\usetheme{metropolis}
\usecolortheme{beaver}

\title{Yang-Mills Instantons}
\author{Seongmin Kim, Taeyoon Kim}
\institute{SNU}
\date{\today}

\begin{document}

\frame{\titlepage}

%%copilot should give 기본 구조 of each page, \begin{frame} \frametitle{} \begin{itemize} \item \end{itemize} \end{frame}
\begin{frame}
\frametitle{Table of Contents}
%% 1. What is instanton?
%% 2. Instanton effect in Path Integral
%% 3. Yang-Mills Instanton : topology, and Anomalies
%% 4. Effects of Instantons in real world
%% 5. Calculation of Instanton effects
%% 6. Conclusion
\begin{itemize}
\item What is instanton?
\item Instanton effect in Path Integral
\item Yang-Mills Instanton : topology, and Anomalies
\item Effects of Instantons in real world
\item Calculation of Instanton effects
\item Conclusion
\item References
\end{itemize}
\end{frame}

\begin{frame}
\frametitle{What is instanton?}
\begin{itemize}
\item Mathematicaly, Instanton is a solution to the classical field equations of motion in Euclidean space.
\begin{equation}
    \delta S_E [\psi_{\text{inst}}] = 0 \rightarrow \text{Euler-Lagrange equation}
\end{equation}
\item Instanton solutions of Euclidean EL equation are localized in (Euclidean) space and time, and have finite (Euclidean) action.
\item In Minkofski QFT, it gives a non-perturbative effect: the tunneling between the classical vacua.
\item Instantons are important in understanding non-perturbative effects and tunneling between vacua in quantum field theory.
\end{itemize}
\end{frame}

\begin{frame}
\frametitle{Instanton effect in Path Integral}
\begin{itemize}
\item In path integral formalism, instantons are classical solution, i.e., the saddle points of the action.
\item Instantons only show up in the Euclidean path integral.
\item Is it physically meaningful? Do we have to consider this effect in Minkowski space QFT?
\item So the path integral is dominated by the instanton contributions in the semiclassical limit.
\item The full path integral is given by the sum of all instanton contributions, with small fluctuations around them.
\end{itemize}
\begin{equation}
Z = \int \mathcal{D}\phi e^{-S[\phi]} = \sum_{\text{instantons}}  e^{-S_{\text{inst}}} \int \mathcal{D}\delta\phi e^{-S_{\text{fluct}}[\delta\phi]}
\end{equation}
\end{frame}
\begin{frame}
    \frametitle{Instanton effect in Path Integral : Example}
    \begin{itemize}
    \item A simple example: the one-dimensional quantum mechanics with two minima.
    \item The potential is given by $V(x) = \frac{1}{2}x^4 - \frac{1}{2}x^2$, and the action is $S = \int dt \left(\frac{1}{2}\dot{x}^2 - V(x)\right)$.
    \item The Euclidean action is $S_E = \int d\tau \left(\frac{1}{2}\dot{x}^2 + V(x)\right)$, and the Euclidean EOM has a classical solution, starting from $x = -1$ at $\tau = -\infty$ and ending at $x = 1$ at $\tau = \infty$.
    \item To evaluate correct path integral, We need to consider all the posible semiclassical paths, including the instanton solution. for example,
    \item If  $T \gg t_{\text{inst}}, $all the instanton solution can be approximated as a composition of $n$ single instanton solution, at time $t_1, ... t_n$, and the instanton action is given by $S = \sum_{i=1}^n S_{\text{inst}}=nS_{\text{inst}}$.

    \end{itemize}
    %% the matrix element of e^(-HT) from x = -1 to x = 1, using the path integral including instanton solution.
    %% small font size
    
\end{frame}
\begin{frame}
    \frametitle{Instanton effect in Path Integral : Calculation}

    So, now we can calculate the matrix element of $e^{-HT}$ from $x = -1$ to $x = 1$, using the path integral including instanton solution.
    \begin{align*}
    \small
    \langle 1 | e^{-HT} | 1 \rangle = \int \mathcal{D}x e^{-S_E} = \sum_{\text{instantons}} e^{-S_{\text{inst}}} \int \mathcal{D}\delta x e^{-S_{\text{fluct}}} \\
    = e^{-T/2} \sum_{n \text{ odd}} \int_{-T/2}^{T/2} d\tau_1 \int_{-T/2}^{\tau_1} d\tau_2 \cdots \int_{-T/2}^{\tau_{n-1}} d\tau_n e^{-nS_{\text{inst}}} \int \mathcal{D}\delta x e^{-S_{\text{fluct}}} 
    \end{align*}


Here, the fluctuation path integral can be factorized into $n$ parts, which are the fluctuation path integrals around each instanton solution.
\begin{align*}
\small
\int \mathcal{D}\delta x e^{-S_{\text{fluct}}} &= \left(\int \mathcal{D}\delta x_n e^{-S_{\text{fluct}}}\right)^n e^{-T/2}= K^n e^{-T/2} \\
    \langle 1 | e^{-HT} | 1 \rangle &= e^{-T/2} \sum_{n \text{ odd}} \int_{-T/2}^{T/2} d\tau_1 \int_{-T/2}^{\tau_1} d\tau_2 \cdots \int_{-T/2}^{\tau_{n-1}} d\tau_n (Ke^{-S_{\text{inst}}})^n \\
    \langle 1 | e^{-HT} | 1 \rangle &=e^{-T/2} \sum_{n \text{ odd}} \frac{1}{n!} K^n(e^{-S_{\text{inst}}})^n T^n = e^{-T/2} \sinh(KTe^{-S_{\text{inst}}}) \\
\end{align*}
\end{frame}

\begin{frame}
\frametitle{Instanton effect in Path Integral : Conclusion}
\begin{itemize}
\item The Path integral including instanton solution gives the correct matrix element of $e^{-HT}$ from $x = -1$ to $x = 1$: $\langle 1 | e^{-HT} | 1 \rangle = e^{-T/2} \sinh(KTe^{-S_{\text{inst}}})$.
\item By changing the result to $T \rightarrow it$, we can get the propagator of the Minkowski QM: $\langle 1 | e^{-iHT} | 1 \rangle = e^{-it/2} \sin(Kte^{-S_{\text{inst}}})$.
\item No instanton solution in Minkowski space, but the instanton effect is still important in understanding the non-perturbative effects in Minkowski QFT.
\item Why? Wick rotation is essential in the path integral formalism, so even if the instanton solution is not a solution of the Minkowski EOM, it still contributes to the path integral.

\end{itemize}
\end{frame}
%new page
\begin{frame}
\frametitle{Yang-Mills Instanton : Introduction}
\begin{itemize}
\item Yang-Mills instantons are the instanton solutions of the Yang-Mills theory.
\item The Yang-Mills action is given by $S = \int d^4x \left(-\frac{1}{4g^2}F_{\mu\nu}^a F^{\mu\nu a}\right)$, and the Euclidean action is $S_E = \int d^4x \left(\frac{1}{4g^2}F_{\mu\nu}^a F^{\mu\nu a}\right)$.
\item The Euclidean EOM is given by $\mathcal{D}_\mu F^{\mu\nu a} = 0$, with the Bianchi identity $\sum_{\text{cyc}}\mathcal{D}_\mu \tilde{F}^{\nu\rho a} = 0$.
\item The YM instanton solution is nontrivial solution of the EOM, and can be obtained by solving the self-dual equation $\tilde{F}^{\mu\nu a} = F^{\mu\nu a}$.
%% 공변미분 D 글씨체 \mathcal{D}
\end{itemize}
\end{frame}

\begin{frame}
    \frametitle{Yang-Mills Instanton : Self-dual equation}
    \begin{itemize}
    \item Hodge dual operator : $\tilde{F}^{\mu\nu a} = \frac{1}{2}\epsilon^{\mu\nu\rho\sigma}F_{\rho\sigma}^a$.
    \item The self-dual equation is given by $\tilde{F}^{\mu\nu a} = F^{\mu\nu a}$.
    \item All the 2-forms can be decomposed into self-dual and anti-self-dual parts: $F^{\mu\nu a} = F^{\mu\nu a}_+ + F^{\mu\nu a}_-$.
    \item If either $F^{\mu\nu a}_+$ or $F^{\mu\nu a}_-$ is zero, the solution is called self-dual or anti-self-dual, and satisfies the EOM of the YM theory, $\mathcal{D}_\mu F^{\mu\nu a} = 0$.
    %% 공변미분 D 글씨체 \mathcal{D}
    \end{itemize}
    \end{frame}

\begin{frame}
        \frametitle{Yang-Mills Instanton : evaluating $S_{\text{inst}}$}
        \begin{itemize}
        \item The instanton action is given by $S_{\text{inst}} = \int d^4x \left(\frac{1}{4g^2}F_{\mu\nu}^a F^{\mu\nu a}\right)$.
        \item The action can be evaluated easily when the instanton is self-dual or anti-self-dual.
        \end{itemize}
        \begin{align*}
        S_{\text{YM}} &= \int d^4x \left(\frac{1}{4g^2}F_{\mu\nu}^a F^{\mu\nu a}\right) \\
        &=  \frac{1}{4g^2} \int d^4x \left(F_{\mu\nu}^a \pm \tilde{F}_{\mu\nu}^a\right)^2 \mp \frac{1}{2g^2} \int d^4x \tilde{F}_{\mu\nu}^a F^{\mu\nu a} \\ 
        &\geq \pm \frac{1}{2g^2} \int d^4x \tilde{F}_{\mu\nu}^a F^{\mu\nu a} = \frac{8\pi^2}{g^2}|n|
        \end{align*}
        So, the instanton action is the bound, $8\pi^2/g^2|n|$, where $n$ is the instanton number(the 2nd chern number).
        This derivation of the instanton action shows the topological nature of the instanton solution : the instanton action is quantized, and the instanton number is a topological invariant of the corresponding $SU(N)$ principal bundle.
        \end{frame}

\begin{frame}
\frametitle{Yang-Mills Instanton : Physical effects}
\begin{itemize}
\item The instanton solution is a non-perturbative effect in the YM theory : $e^{-S_{\text{inst}}}=e^{-8\pi^2/g^2|n|}$ is the tunneling amplitude between the classical vacua, and non-perturbative in $g$.
\item The instanton is a source of tunneling between classical vacua in the YM theory, and the true vaccume structure of the YM theory, and the $\theta$ term.
\item THe instanton solution is also important in understanding the anomalies in the YM theory.
\end{itemize}
\end{frame}
\end{document}
