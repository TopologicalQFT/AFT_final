\documentclass{article}
\usepackage{tikz-cd}
\usepackage{amsmath}
\usepackage{amsfonts}
\usepackage{graphicx} % Required for inserting images
\usepackage{amsthm}
\usepackage[scr=rsfs]{mathalpha}

\newtheorem{defn}{Definition}
\newtheorem{thm}{Theorem}
\newtheorem{lem}{Lemma}
\newtheorem{prop}{Proposition}
\newtheorem{rem}{Remark}
\newtheorem{note}{Note}
\newtheorem*{exa}{Ex)}



\title{Instanton}
\author{Taeyoon Kim}
\date{}

\begin{document}
\maketitle

\section*{Yang Mills Theory}

\begin{rem}[Exterior Covariant derivative]
    One can define the exterior covariant derivative. 
    \[
        d_\nabla: \Omega^k(M,E)\rightarrow \Omega^{k+1}(M,E)
    \]
    With the following properties. For $s\in \Gamma(M,E)$,
    \[
        d_\nabla s = \nabla s 
    \]
    For $\omega\in \Omega^k(M,E)$, $\eta \in \Omega^l(M,E)$
    \[
        d_\nabla(\omega \wedge \eta ) = d_\nabla \omega \wedge \eta + (-1)^k \omega \wedge d_\nabla \eta
    \]
\end{rem}

Note that this exterior covariant derivative gets along well with our former calculation. 

\begin{note}
    Select a local frame of $E$ as $\{e_1\cdots e_r\}$. 
    $\omega = \sum \omega^i\otimes e_i$, $\omega_i \in \Omega^k(M)$. One can calculate the exterior covariant derivative.
    \[
        d_\nabla \omega = d_\nabla e_i \wedge \omega_i + e_i \otimes d\omega_i = A^j_i e_j \wedge \omega^i + e_i \otimes d\omega^i = e_j \otimes (d\omega^j + A^j_i \wedge \omega^i)
    \]
    For some covariant derivative matrix $A\in\Omega^1(M,\mathrm{End} E)$ for local frames. 
    \[
        d_\nabla \omega = d\omega + A\wedge \omega
    \]
\end{note}

Since it is crucial to calculate covariant derivative for endomorphism valued forms, one needs to define the following covariant derivative. 

\begin{rem}[Covariant Derivative on $\mathrm{End}E$]
    For $\eta\in \Gamma(M,\mathrm{End}E)$, $s\in \Gamma(M,E)$, the covariant derivative of $\eta$ is defined as the following. 
    \[
        \nabla^{\mathrm{End}E}_X(\eta) (s) = \nabla_X^E(\eta s) - \eta \nabla_X^E(s)
    \]
    Which inherits the property of the Leibniz rule. 
\end{rem}

It is valid to define new exterior covariant derivative for endomorphism valued forms. 

\begin{prop}[Exterior Covariant Derivative for Endomorphism]
    Exterior Covariant derivative for endomorphism also must have the following property.
    For $\eta\in\Omega^k(M,\mathrm{End}E)$, $\omega \in \Omega^l(M,E)$,
    \[
        d_\nabla^E(\eta\wedge\omega) = d_\nabla^{\mathrm{End}E}(\eta) \wedge \omega+ (-1)^k \eta\wedge d_\nabla^E(\omega)
    \]
\end{prop}
\begin{proof}
    Rewrite that $\eta = A\otimes \alpha$, $\omega = s \otimes \beta$ For $\alpha,\beta\in\Omega^*(M)$.
    \[
        d_\nabla^E(A(s)\otimes\alpha\wedge\beta) = \nabla^E(A(s))\otimes\alpha\wedge\beta + A(s)d_\nabla^E(\alpha\wedge\beta)
    \]
    \[
        =(\nabla^{\mathrm{End}E}A(s)+A(\nabla^Es))\otimes \alpha \wedge \beta + A(s)(d\alpha\wedge\beta +(-1)^k\alpha\wedge d\beta)
    \]
    \[
        =(\nabla^{\mathrm{End}E}A\wedge\alpha+A\otimes\alpha)\wedge(s\otimes\beta) + (-1)^k(A\otimes\alpha)(\nabla^Es\otimes\beta + s\otimes d\beta)
    \]
    \[
        = d_\nabla^{\mathrm{End}E}(A\otimes\alpha) \wedge s\otimes\beta+ (-1)^k A\otimes \alpha\wedge d_\nabla^E(s\otimes\beta)
    \]
\end{proof}
By this calculation, we can assure that endomorphism valued forms admit exterior covariant derivative. And now we omit the $\mathrm{End}E$ for such symbols. \\
For a vector bundle $E$ over $M$, one can define the Yang-Mills action defined on the space of connections of $E$.

\begin{defn}[Yang-Mills Action] 
\[
    S_{YM}(\nabla) = \int_M \frac{1}{2g^2} \mathrm{tr}(F_\nabla\wedge\star F_\nabla) = \int_M \frac{1}{2g^2} \langle F_\nabla, F_\nabla \rangle \mathrm{Vol}_g
\]
In coordinate basis, 
\[
    S_{YM}(\nabla) = \int_M d^4 x \frac{1}{2g^2} \mathrm{tr}(F^{\mu\nu}F_{\mu \nu})
\] 
While $F$ is endomorphism valued 2 forms on M. 
\end{defn}
More physically, one can make a same description for principal G bundle over M. The associated vector bundle(i.e. its lie algebra bundle) would give the same construction on Yang-Mills Action. It is valid to consider its saddle point, checking classic solutions. This is the natural reason why one has the $\mathrm{tr}$ for the action, since the natural metric for (semi-) simple lie algebra is the Killing form. 

\begin{prop}[Yang-Mills Equation of motion]
     $F$ that lies on the critical point of the action functional satisfies the equation, $d_\nabla^{\star}F=0$, while $d_\nabla^\star$ represents the adjoint of $d_\nabla$. 
\end{prop}
\begin{proof}  
    Consider the following $\nabla_t = \nabla + tA$ for arbitrary $A$. Which yields the t-dependent $F_t = F_\nabla + t d_\nabla A + \frac{t^2}{2} [A,A]$
    \[
        \left. \frac{d}{dt} \right|_{t=0}  S_{YM}(\nabla+tA) = \int_M \frac{1}{g^2} \mathrm{Vol}_g \langle F_\nabla, d_\nabla A\rangle = \int_M \frac{1}{g^2} \mathrm{Vol}_g \langle d_\nabla ^\star F_\nabla, A\rangle =0
    \]

    Giving that $d_\nabla ^\star F = 0$.
\end{proof}


\begin{note}
    One can calculate $d_\nabla d_\nabla \omega$ for $\omega\in\Omega^k(M,E)$.
    \[
        d_\nabla d_\nabla\omega = d_\nabla(d\omega+A\wedge\omega) = d(A\wedge\omega)+A\wedge(d\omega + A\wedge\omega)
    \]
    \[
        =(dA+A\wedge A)\omega = F_\nabla \wedge \omega
    \]
\end{note}


\begin{rem}
    The Bianchi identity, $d_\nabla F = 0$ must be satisfied for any endomorphism valued 2 forms. 
\end{rem}
\begin{proof}
    For any $\omega\in\Omega^1(M,E)$,
    \[
    (d_\nabla)^3\omega = d_\nabla(F_\nabla\wedge\omega) = d_\nabla F_\nabla\wedge\omega + F_\nabla \wedge d_\nabla\omega
    \]
    \[
        =(d_\nabla)^2 d_\nabla\omega = F_\nabla \wedge d_\nabla \omega
    \]

    Therefore we have $d_\nabla F_\nabla=0$
\end{proof}

Through this process, one can derive PDE that $F_\nabla$ must satisfy for classic Yang Mills gauge theory. To simplify, 
\[
    d_\nabla^\star F_\nabla = 0, d_\nabla F_\nabla = 0
\]

\section*{Yang Mills Instanton}

\begin{defn}[Hodge star operator]
    One can define the following Hodge star operator for n-dimensional Riemannian manifold $(M,g)$.
    \[
     \star : \Omega^k(M)\longrightarrow \Omega^{n-k}(M)
    \]
    such that for any $\eta\in \Omega^k(M)$,
    \[
        \langle\eta,\omega\rangle \mathrm{Vol}g =\eta \wedge \star \omega
    \]
    While the $\mathrm{Vol}g$ represents the volume form of the Riemannian manifold $M$.
\end{defn}

\begin{note}[Hodge star operator for index notation]
    Choosing the coordinate chart $(U,\phi = (x^1,\cdots ,x^n))$ for $M$, one can explicitly write down the Hodge star operator for k forms.
    \[
        \star (dx^{i_1}\wedge \cdots \wedge dx^{i_k}) = \frac{\sqrt{\mathrm{det}g_{ij}}}{(n-k)!} g^{i_1 j_1}\cdots g^{i_k j_k} \epsilon_{j_1\cdots j_n} dx^{j_{k+1}}\wedge\cdots\wedge dx^{j_n}
    \]
\end{note}
\begin{note}
    One can calculate the following $\star\star:\Omega^k(M)\rightarrow\Omega^k(M)$.
    And it is known that $\star\star\omega = (-1)^{k(n-k)}\omega$. And for $n=4$, $k=2$, $\star\star = \mathrm{id}$  on $\Omega^2(M)$.
\end{note}
While it is natural to define such star operator on vector valued or endomorphism valued forms, to just apply it on the differential form part. Therefore, one can define the following.
\begin{defn}[Yang Mills Instanton]
    $F\in\Omega^2(M,\mathrm{End}E)$ that satisfies
    \[
        \star F = \pm F
    \]
    are called self dual or anti self dual curvature form on M, and they are called the instanton solution of the Yang Mills equation. 
\end{defn}

Note that one can  always decompose $F$ into self dual and anti self dual parts.
\[
    F  = \frac{ F + \star F}{2} + \frac{ F - \star F}{2} = F_+ + F_-
\]

\begin{lem}
    $d_\nabla^\star = (-1)^k \star^{-1} d\star $ on $\Omega^k(M)$
\end{lem}
\begin{proof}
   \[
        \langle d\omega,\eta \rangle = \int_M d\omega \wedge \star \eta = \int_M d(\omega\wedge\star\eta) - (-1)^{k-1}\omega \wedge d\star\eta = \int_M (-1)^k \omega \wedge d\star \eta
     \]   
    \[= \int_M \omega\wedge \star \star^{-1} (-1)^k d\star\eta = \langle\omega,(-1)^k\star^{-1}d\star \eta\rangle
   \]
   
\end{proof}

\begin{note}
    (Anti-)Self dual $F$ are the solutions for the Yang Mills equation. The Bianchi identity assures that $d_\nabla F_\nabla =0$. While $d_\nabla^\star F_\nabla = d_\nabla \star d_\nabla \star F_\nabla = \pm d_\nabla \star d_\nabla F_\nabla = 0$ for Instanton solutions. 
\end{note}

In fact, the Instanton solutions are the absolute minima of the Yang Mills action. This physically means that Instanton solutions are the vacuum solution for the Yang Mills equation. 
\begin{prop}[Instanton as the Absolute minima]
    YM Instanton is the absolute minima of the YM action.
\end{prop}
\begin{proof}
    \[
        S_{YM}(\nabla) = \frac{1}{2g^2}\int_M \mathrm{tr}(F_\nabla\wedge \star F_\nabla) = \frac{1}{2g^2}\int_M \mathrm{tr}((F_+ + F_-)\wedge \star (F_+ + F_-))
    \]
    \[
        \frac{1}{2g^2}\int_M \mathrm{tr}((F_+ + F_-)\wedge (F_+ - F_-)) = \frac{1}{2g^2}\left(||F_+||^2 + ||F_-||^2 \right)
    \]
    While 
    \[
        c(\nabla) = \frac{1}{2g^2}\int_M \mathrm{tr}(F_\nabla\wedge F_\nabla) = \frac{1}{2g^2}\left(||F_+||^2 - ||F_-||^2 \right)
    \]
    This assures that
    \[
        |c(\nabla)| \leq S_{YM}(\nabla)
    \]
    And the equality holds when $F=F_{\pm}, F_{\mp}=0$
\end{proof}

The additional information from the former proposition is the value of the absolute minimum. Where the $c(\nabla)$ is the 2nd Chern class of the connection $\nabla$. One can actually give the calculation so that
\[
    c(\nabla) = \frac{8\pi^2}{g^2} n
\]
for some integer $n\in\mathbb{Z}$.

\begin{rem}[Chern class has integral coefficients]
    Chern class is the Characteristic class, which must naturally related to the cohomology of integral coefficients. 
\end{rem}
Characteristic class is the functor from $[-,\mathrm{BG}]$ (i.e. the homotopy classes of maps from space to the classifying space of the Lie group G) to $H^*(-,\mathbb{Z})$. While one knows the categorical equivalences between 2 groupoids, $\mathrm{Bun}_G(X)/\sim$ and $[-,\mathrm{BG}]$. While the $\mathrm{Bun}_G(X)/\sim$ represents the category of principal $G$ bundles over $X$ modulo gauge transformations. This gives us the following diagram.
\[
\begin{tikzcd}
   \mathrm{Bun}_G(-)/\sim \arrow[r] 
    & {[-,\mathrm{BG}]}\arrow[l] \arrow[d, "Ch.Class"] \\
     & {H^*(-,\mathbb{Z})}
\end{tikzcd}
\]

If we are given a data of $\mathrm{Bun}_G(-)/\sim$, there exists a way to have some $H^*(-,\mathbb{R})$. Which is known as the Chern-Weil homomorphism. The only remaining step is to check that such object made by Chern Weil is the characteristic class. The mathematical definition of the Chern class has some axioms, and one can verify if the Chern-Weil satisfies the axioms. While the Chern class can give us the integrality of $U(N)$ bundle, it is possible to explain the integrality of the subgroups of $U(N)$. 

\section*{Moduli space of YM Instanton}
As a sense that Instanton is the classic solution to the YM equation, it is crucial to investigate such solutions. Atiyah-Hitchin-Singer (1977,1978) showed that the moduli space of Instanton, is a finite dimensional manifold for a certain condition. Such condition is for the base 4-manifold, and We will simply review how to calculate the dimension of the moduli space.

\begin{thm}[Atiyah-Hitchin-Singer]
    Let $M$ be a compact, self-dual Riemannian manifold with positive scalar curvature. Let $P$ be a principal $G$-bundle over $M$, where $G$ is a compact semisimple Lie group. Then the space of moduli of irreducible self-dual connections is either empty or a manifold of dimension 
    \[
        p_1(\mathfrak{g})-\frac{1}{2}G(\chi-\tau)
    \]
    While $\chi$ is the Euler characteristic of $M$ and $\tau$ is the signature of $M$. 
\end{thm}

First is to understand backgrounds for the theorem. For simplicity, we only think of the connected manifolds.

\begin{defn}[Holonomy Group]
    For a given connection $\nabla$, with a loop $\gamma:[0,1]\rightarrow M$ that $\gamma(0)=\gamma(1)=p$, the holonomy $P_\gamma(\nabla)$ is a parallel transport around the loop. 
\end{defn}

Such parallel transport is defined to be $\mathrm{End}E$, in a vector bundle setting. While in the principal bundle setting, such parallel transport can be understood as the transformation on fiber. (This can be proven by choosing the chart, and making it into a problem of ODE.) This means that the parallel transport is not the problem about vectors, but about the Lie group. 

\begin{prop}[Gauge invariance of Prallel transport]
    For any gauge transformation $\Phi:M\rightarrow G$, the parallel transformation of $\nabla$ through a curve $\gamma$ with $\Phi$, can be written as the following.
    \[
        P^{\Phi}_\gamma = \Phi(\gamma(1)) P_\gamma \Phi(\gamma(0))^{-1}
    \]
\end{prop}

Which is trivial from the fact that parallel transport on principal G bundle is the transform on fiber. 

\begin{defn}[Irreducible Connection]
\end{defn}

The irreducibility of connection allows us to claim that gauge transformation acts freely on the space of connections. One knows that all Holonomy groups for each points of $M$, are isomorphic. Using the gauge invariance of the parallel transport,
\[
    P^{\Phi}_\gamma(x) = \Phi^{-1}(x) P_{\gamma}(x) \Phi(x)
\]
If $\Phi$ leaves the connection, i.e. if it lies on the kernel of the action, 
\[
    P^{\Phi}_\gamma(x) = P_\gamma(x) = \Phi^{-1}(x) P_{\gamma}(x) \Phi(x)
\]
Which claims that $\Phi(x)$ is in the center of $G$. Since ant $g\in G$ can be realized by some $\gamma$. For groups like $\mathrm{U}(N)$ or $\mathrm{SU}(N)$, it is trivial that such center is $1$, up to scalar. Meaning that it acts (almost) freely. This would play an important role in calculating the dimension of the Instanton (Anti-) self dual moduli space. Moreover, it is known that any semisimple Lie group admits a 0-dimensional center. 

\begin{proof}[Sketch of proof for Thm 1]
    The main is to investigate the tangent space of the manifold $T_A \mathscr{A}_{+}/\mathscr{G}$, the self dual Instanton modulo gauge transformation.
    \[
    \begin{tikzcd}
        0 \arrow[r] & \Omega^0(M, \mathfrak{g}) \arrow[r, "d_\nabla"] & \Omega^1(M, \mathfrak{g}) \arrow[r,"d_\nabla^{-}"] & \Omega^2_{-}(M, \mathfrak{g}) \arrow[r] &  0
    \end{tikzcd}
    \]
    The first claim is that if $\tau \in T_A\mathscr{A}/\mathscr{G}$ then $\tau\in\mathrm{Im}d$. Consider the following gauge transformation. Taking the local chart and for any lie algebra $X\in\mathfrak{g}$, 
    \[
        A(t) = e^{-tX}A e^{tX} + e^{-tX}d e^{tX}
    \]
    And the differential of the map, i.e. the tangent connection at $A$, is as the following.
    \[
        \left.\frac{d}{dt}\right|_{t=0}A(t)= [X,A] - dX = -d_\nabla X
    \]
    This proves that the element of $T_A\mathscr{A}/\mathscr{G}$ lies in $\mathrm{Im}d_\nabla$. The second is that the element of $T_A\mathscr{A}_+$ lies at the kernel of 2nd map. Which is trivial by definition. Therefore our main interest reduces to calculating the first cohomology of the complex. \\
    From here, it highly depends on the original paper of Atiyah-Hitchin-Singer. Witzenböck positivity argument shows that the 2nd cohomology must vanish. 
        For a covariant exterior derivative $d_\nabla$, 
        \[
            \Delta= d_\nabla^{-} (d_\nabla^{-})^\star = \frac{1}{2}d_\nabla d_\nabla^\star + \frac{R}{6} - W_-
        \]
        $R$ represents the Ricci scalar, and $W_-$ represents the anti self dual part of Weyl tensor.
    In our setting, this shows that $d_\nabla^-$ is positive-definite and hence any $\omega\in H^2_-(M,\mathfrak{g})$ is $d_\nabla^\star\omega$ up to positive constant. Which proves that $h_2=0$.\\
    The 1st cohomology is the kernel of $d_\nabla$. For any lie algebra $X$,
    \[
        -\left.\frac{d}{dt}\right|_{t=0}A(t)= d_\nabla X
    \]
    And this would give that $A$ is invariant under $e^{tX}$, which we know that the irreducibility of the bundle shows $X=0$. 
    \\
    For compact $M$, it is known that the complex is elliptic, and hence it index equals the topological index by Atiyah-Singer index theorem. To calculate its analytic index, one needs to define the following Dirac operator.
    \[
       D = d_\nabla^{\star} + d_\nabla^{-}:\Omega^1(M,\mathfrak{g})\rightarrow \Omega^0(M,\mathfrak{g})\oplus\Omega^2_{-}(M,\mathfrak{g})
    \]
    Using the Atiyah-Singer Index theorem, 
    \[
\mathrm{ind}D = \int_M \mathrm{ch} \left( \bigoplus_i (-1)^i E_i \right) \frac{\mathrm{Td}(TM \otimes_\mathbb{R} \mathbb{C})}{\mathrm{e}(TM)}
\]
The $E_i$s are given as the following.
\[
    E_0 = \mathfrak{g}\otimes \Omega^0(M)_{\mathbb{C}}
\]
\[
    E_1 = \mathfrak{g}\otimes \Omega^1(M)_{\mathbb{C}}
\]
\[
    E_2 = \mathfrak{g}\otimes \Omega^2_{-}(M)_{\mathbb{C}}
\]
Since the Chern character is splitted by tensoring, 
\[
    \mathrm{ch} \left( \bigoplus_i (-1)^i E_i \right)
    = \mathrm{ch}(\mathfrak{g})\mathrm{ch}(\Omega^0(M)-\Omega^1(M)+\Omega^2_{-}(M))_\mathbb{C}
\]
The Chern character of $\mathfrak{g}$ would be $\mathrm{dim}G+\frac{1}{2}p_1(\mathfrak{g})$, while the remaining terms would give $b_0 - b_1 + b_2^{-}$ for the 4 form part. While the zero form part is highly nontrivial, which is known as 2. Through such discussion, the index would be $\mathrm{dim}G (b_0-b_1+b_2)+p_1(\mathfrak{g})$. Through some calculation, one can know that it is $p_1(\mathfrak{g})+\frac{1}{2}\mathrm{dim}G(\chi-\tau)$
\end{proof}
Note that Atiyah-Hitchin-Singer's original paper uses an equivalent elliptic operator on Spin bundle, through an isomorphism between spin bundle and bundle of differential forms. 

\begin{prop}
    $\mathrm{SU}(2)$ instanton moduli over $S^4$ is $8k-3$ dimensional.
\end{prop}
\begin{proof}
    One knows that $\chi(S^4)=2$, $\tau(S^4)=0$. The pontryagin class for $\mathfrak{su}(2)$ can be constructed from the product bundle, $E\otimes E$. Where $E$ is the 2 dimesional representation of $\mathrm{SU}(2)$, vector bundle of $c_2(E)=-k$. Therefore one can consider that $E\otimes E = S^2E\oplus \wedge^2 E$. The former represents the symmetric part, while the latter represents the antisymmetric part. The antisymmetric part is of rank 2 choose 2, therefore a line bundle. While the symmetric part is of a rank 3. Taking the chern character would give that,
    \[
        (2+c_2(E))^2 = 3 + \frac{1}{2}p_1(\mathfrak{su}(2))+1 
    \]
    Considering only 4 dimensional part, one would have that $p_1(\mathfrak{su}(2))=8k$ and hence the total dimension is $8k-3$.
\end{proof}


\section*{BPST Instanton}
There exists a method to construct instanton in $\mathbb{R}^4$. One can  have the ansatz
\[
A_{\mu}(x) = \alpha \sigma_{\mu \nu} \partial_{\nu} \log \phi(x^2).
\]
While $\sigma_{\mu\nu}$ are generators for rotation in $\mathbb{R}^4$,satisfying
\[
    [\sigma_{\mu\nu},\sigma_{\rho\sigma}] = -2\left(\delta_{\mu \rho} \sigma_{\nu \sigma}+\delta_{\nu \sigma} \sigma_{\mu \rho}-\delta_{\mu \sigma} \sigma_{\nu \rho}-\delta_{\nu \rho} \sigma_{\mu \sigma}\right)
\]
Then the field strength and dual field strength are:
\[
F_{\mu \nu} = \alpha \sigma_{\nu \rho} \partial_{\mu} \partial_{\rho} \log \phi 
              - \alpha \sigma_{\mu \rho} \partial_{\nu} \partial_{\rho} \log \phi^2 
              + \alpha^2 [\sigma_{\mu \rho}, \sigma_{\nu \sigma}] (\partial_{\rho} \log \phi)(\partial_{\sigma} \log \phi)
\]
\[
           = \left(\alpha \sigma_{\nu \rho} \partial_{\mu} \partial_{\rho} \log \phi 
              - (\mu \leftrightarrow \nu) \right) 
              + 2 \alpha^2 \left( \sigma_{\mu \sigma} (\partial_{\nu} \log \phi) (\partial_{\sigma} \log \phi) 
              - (\mu \leftrightarrow \nu) \right) 
              - 2 \alpha^2 \sigma_{\mu \nu} (\partial \log \phi)^2;
\]
\[
*F_{\mu \nu} = \frac{1}{2} \epsilon_{\mu \nu \rho \sigma} F_{\rho \sigma}
\]
\[
           = \alpha \epsilon_{\mu \nu \rho \sigma} \sigma_{\nu \lambda} \partial_{\rho} \partial_{\lambda} \log \phi 
              + 2 \alpha^2 \epsilon_{\mu \nu \rho \sigma} \sigma_{\rho \lambda} (\partial_{\sigma} \log \phi) (\partial_{\lambda} \log \phi) 
              - \alpha^2 \epsilon_{\mu \nu \rho \sigma} \sigma_{\rho \sigma} (\partial \log \phi)^2
\]
\[
           = \sigma_{\nu \rho} \left( \alpha \partial_{\rho} \partial_{\mu} \log \phi 
              - 2 \alpha^2 (\partial_{\rho} \log \phi) (\partial_{\mu} \log \phi) 
              - (\mu \leftrightarrow \nu) \right) 
              + \sigma_{\mu \nu} (\alpha \partial^2 \log \phi).
\]

To equip self duality, $-2\alpha^2(\partial \mathrm{log}\phi)^2 = \alpha\partial^2\phi$ needs to be satisfied. Substituting $\phi\rightarrow\phi^{1/2\alpha}$, one can get that $\phi^{-1}\partial^2\phi=0$, which is just laplace equation.\\
 For $\phi = \frac{\rho^2}{(x-a)^2}+C$, C needs to be 1 for a condition that sufficiently large x gives 0 on $A$.
\[
    A_\mu(x) = -\sigma_{\mu\nu}\frac{\rho^2(x-a)_\nu}{(x-a)^2((x-a)^2+\rho^2)}
\]

Under $U=\frac{i\sigma_i x_i}{|x|}$, 
\[
    U(\partial_\mu +A_\mu )U^{-1} =  -\bar{\sigma}_{\mu\nu}\frac{(x-a)_\nu}{((x-a)^2+\rho^2)}
\]
The field strength would be followed.
\[
    F_{\mu \nu}=  \partial_\mu\left[-\bar{\sigma}_{\nu \rho} \frac{(x-a)_\nu}{(x-a)^2+\rho^2}\right]-(\mu \leftrightarrow \nu)+\frac{(x-a)_\rho(x-a)_\sigma}{\left[(x-a)^2+\rho^2\right]^2}\left[\bar{\sigma}_{\mu \rho}, \bar{\sigma}_{\nu \sigma}\right] 
\]
\[=  {\left[\frac{\bar{\sigma}_{\mu \nu}}{(x-a)^2+\rho^2}+\frac{2(x-a)_\mu(x-a)_\rho}{\left[(x-a)^2+\rho^2\right]^2} \bar{\sigma}_{\nu \rho}\right]-(\mu \leftrightarrow \nu) } 
\]
\[ -\frac{2}{\left[(x-a)^2+\rho^2\right]^2}\left((x-a)^2 \bar{\sigma}_{\mu \nu}-(x-a)_\rho(x-a)_\mu \bar{\sigma}_{\rho \nu}-(x-a)_\nu(x-a)_\rho \bar{\sigma}_{\mu \sigma}\right) 
\]
\[  =\frac{2 \rho^2 \bar{\sigma}_{\mu \nu}}{\left[(x-a)^2+\rho^2\right]^2}\]

    each $a$ corresponds to the translational transformation of instanton, with $\rho$ a sizing transformation. These are the conformal transformations, making the instanton valid. 
The winding number(i.e. the 2nd chern class integrated)
\[
    k=-\frac{1}{16 \pi^2} \int \mathrm{d}^4 x \operatorname{tr} F_{\mu \nu}^* F_{\mu \nu}=-\frac{1}{16 \pi^2} \int \mathrm{d}^4 x \frac{4 \rho^4}{\left[(x-a)^2+\rho^2\right]^4} \operatorname{tr} \bar{\sigma}_{\mu \nu} \bar{\sigma}_{\mu \nu}=1
\]


These matches well with our observations. Since the dimension of instanton moduli, $\mathrm{SU}(2)$ bundle over $S^4$ is equivalent for instanton moduli over $\mathbb{R}^4$, vanishing at infinity. And such degree of freedom is realized as 5 conformal transformations. While $8k$ is physically intuitive, considering the symmetries of $\mathrm{SU}(2)$, but 3 degrees of freedom are dropped for the gauge transformation.
\\
Therefore, if one doesn't consider the vanishing gauge field, it is possible to consider the gauge configuration on 2 half sphere of $S^4$, intersecting at $S^3$. This would give the matching condition as in the magnetic monopole construction.
\\
As a persepective of pure gauge story, the only information about the instanton at infinities are the winding numbers. Which needs to be true, since the action functional $\mathrm{tr}(F\wedge F)$ is locally exact, as $d\mathrm{tr}(A dA + \frac{2}{3}A^3)$. Meaning that for $S^4$ configuration, the stokes theorem gives the information on boundary, as the Chern Simons form. 
\\
With our former consideration on characteristic class, it is valid to consider the $[S^4,\mathrm{BSU}(2)]=\pi_4(BSU(2))\cong\mathbb{Z}$. While this $\mathbb{Z}$ represents the winding number. In a monoidal perspective, gauge configuration can be multiplied. And such information is in the 2nd chern class. Physically,
\[
    J^\mu = \frac{1}{24\pi^2}\mathrm{tr}(\partial_\nu g g^{-1}\partial_\rho g g^{-1}\partial_\sigma g^{-1})
\]
Is the topological current, counting such instanton numbers. Which can be interpreted in a pure gauge persepective, 
\[
    J \approx \mathrm{tr}(A^3)
\]
Which is asymptotically a Chern Simons term on boundary.


\end{document}